\documentclass[aspectratio=169, 10pt]{beamer}

% --- Pakete & Design ---
\usepackage[utf8]{inputenc}
\usepackage[T1]{fontenc}
\usepackage[ngerman]{babel}
\usepackage{graphicx}
\usepackage{booktabs} % Für schöne Tabellenlinien
\usepackage{hyperref}
\usepackage{xcolor}
\usepackage{tikz}

% Theme
\usetheme{Madrid}
\usecolortheme{whale}
\setbeamertemplate{navigation symbols}{} % Navigationsleiste ausblenden

% Metadaten
\title[SBB Pünktlichkeits-Mining]{Data Mining im Schienenverkehr:\\Pünktlichkeitsanalyse der SBB}
\subtitle{Mythos "Puffer" – Eine datengestützte Untersuchung}
\author[Statistik Projektgruppe]{Statistik Projekt HS25}
\institute[Uni SG]{Universität St. Gallen}
\date{\today}

\begin{document}

% --- Folie 1: Titel ---
\begin{frame}
    \titlepage
\end{frame}

% --- Folie 2: Der Hook ---
\begin{frame}{Die Forschungsfrage}
    \begin{columns}
        \begin{column}{0.6\textwidth}
             Täglich verlassen wir uns darauf, dass das Schweizer Schienennetz wie ein Uhrwerk funktioniert.
             
             \vspace{0.5cm}
             \textbf{Die gängige Annahme:}
             \begin{itemize}
                 \item Grosse Bahnhöfe (Hubs) wie Zürich HB haben lange Haltezeiten.
                 \item Sie sollten als \textit{Puffer} wirken und Verspätungen abbauen.
             \end{itemize}
        \end{column}
        \begin{column}{0.4\textwidth}
            \centering
            % Optional: Ein Symbolbild oder Icon hier
            \Huge \textbf{?}
        \end{column}
    \end{columns}

    \vspace{0.8cm}
    
    \begin{alertblock}{Unsere Kernfrage}
        \centering
        \textbf{Retten uns die grossen Knotenpunkte den Fahrplan,\\oder sind sie eigentlich das Problem?}
    \end{alertblock}
\end{frame}

% --- Folie 3: Datenbasis (Tabelle statt Screenshot) ---
\begin{frame}{Datenbasis: IST-Daten SBB}
    \textbf{Überblick:}
    \begin{itemize}
        \item Zeitraum: September 2025
        \item Umfang: \textbf{> 4.6 Millionen} Zugfahrten (nach Bereinigung)
        \item Tech-Stack: \texttt{Polars} (Python) für High-Performance ETL
    \end{itemize}

    \vspace{0.5cm}
    \textbf{Datenstruktur (Auszug):}
    \begin{table}[]
        \centering
        \small
        \begin{tabular}{lllc}
            \toprule
            \textbf{Datum} & \textbf{Linie} & \textbf{Haltestelle} & \textbf{Verspätung (s)} \\
            \midrule
            01.09.2025 & IR3 & Schaffhausen & 114.0 \\
            01.09.2025 & RE3 & Zürich HB & 42.0 \\
            01.09.2025 & S12 & Winterthur & 12.0 \\
            \dots & \dots & \dots & \dots \\
            \bottomrule
        \end{tabular}
        \caption{Beispielhafte Rohdatenstruktur (Ankunft)}
    \end{table}
\end{frame}

% --- Folie 4: Status Quo (ECDF Plot) ---
\begin{frame}{Status Quo: Wie pünktlich sind wir?}
    \begin{columns}
        \begin{column}{0.4\textwidth}
            \textbf{Die Verteilung:}
            \begin{itemize}
                \item Die Daten sind \textbf{rechtsschief} (nicht normalverteilt).
                \item \textbf{Median Ankunft:} $+0.62$ Min.
                \item \textbf{Long Tail:} Ca. 5.5\% der Züge haben mehr als 3 Min. Verspätung.
            \end{itemize}
        \end{column}
        \begin{column}{0.6\textwidth}
            \centering
            % BILD 1: ECDF Plot
            \includegraphics[width=\textwidth, height=6cm, keepaspectratio]{plot_ecdf.png}
        \end{column}
    \end{columns}
\end{frame}

% --- Folie 5: Zeitliche Muster ---
\begin{frame}{Der "Puls" des Netzes}
    \textbf{Hypothese:} Die Rush-Hour destabilisiert das System.
    
    \begin{figure}
        \centering
        % BILD 2: Tagesverlauf
        \includegraphics[width=0.8\textwidth, height=5.5cm, keepaspectratio]{plot_tagesverlauf.png}
    \end{figure}
    
    \vspace{-0.2cm}
    \footnotesize \textbf{Erkenntnis:} Kritischste Phase ist der Abend (16-19 Uhr). Verspätungen schaukeln sich über den Tag auf.
\end{frame}

% --- Folie 6: Methodik Bahnhof-Mining ---
\begin{frame}{Deep Dive: Das "Bahnhof-Mining"}
    Um die Puffer-Hypothese zu prüfen, berechnen wir für jeden Bahnhof eine neue Metrik:
    
    \begin{block}{Delta Delay ($\Delta$)}
        \[ \Delta = \text{Verspätung}_{\text{Abfahrt}} - \text{Verspätung}_{\text{Ankunft}} \]
    \end{block}
    
    \vspace{0.5cm}
    
    \begin{itemize}
        \item \textcolor{blue}{\textbf{$\Delta < 0$ (Negativ):}} Der Zug fährt pünktlicher ab, als er ankam. \\ $\rightarrow$ \textbf{Zeit gutgemacht (Puffer wirkt).}
        \item \textcolor{red}{\textbf{$\Delta > 0$ (Positiv):}} Der Zug sammelt zusätzliche Verspätung. \\ $\rightarrow$ \textbf{Verspätung aufgebaut.}
    \end{itemize}
\end{frame}

% --- Folie 7: Ergebnisse Scatterplot ---
\begin{frame}{Ergebnisse: Der Mythos fällt}
    \textbf{Korrelation: Bahnhofsgrösse (Anzahl Züge) vs. Delta}
    
    \begin{columns}
        \begin{column}{0.65\textwidth}
            \centering
            % BILD 3: Scatterplot
            \includegraphics[width=\textwidth, height=6cm, keepaspectratio]{plot_bahnhof_scatter.png}
        \end{column}
        \begin{column}{0.35\textwidth}
            \textbf{Befund:}
            \begin{itemize}
                \item Spearman-Korrelation: $\approx \mathbf{0.12}$
                \item Das Vorzeichen ist \textbf{positiv}!
            \end{itemize}
            
            \vspace{0.5cm}
            \textbf{Interpretation:}
             Grosse Bahnhöfe (rechts) liegen nicht im blauen Bereich. Sie bauen Verspätungen eher auf als ab.
        \end{column}
    \end{columns}
\end{frame}

% --- Folie 8: Externe Faktoren (Plot + Tabelle) ---
\begin{frame}{Weitere Einflussfaktoren}
    \begin{columns}
        % Linke Spalte: Stadt/Land (Plot)
        \begin{column}{0.5\textwidth}
            \textbf{1. Stadt vs. Land}
            \begin{center}
                % BILD 4: Urban vs Rural
                \includegraphics[width=\textwidth, height=4cm, keepaspectratio]{plot_urban_rural.png}
            \end{center}
            \footnotesize Unterschied signifikant, aber Effektstärke minimal ($\epsilon^2 < 0.01$).
        \end{column}
        
        % Rechte Spalte: Wochenende (Tabelle)
        \begin{column}{0.5\textwidth}
            \textbf{2. Werktag vs. Wochenende}
            \vspace{0.2cm}
            
            \begin{table}[]
                \centering
                \small
                \begin{tabular}{lrr}
                    \toprule
                    \textbf{Kategorie} & \textbf{Pünktlich} & \textbf{Verspätet} \\
                    \midrule
                    Werktag & 2'632'728 & 213'135 \\
                    Wochenende & 1'283'756 & 80'319 \\
                    \midrule
                    \textbf{Quote} & \textbf{92.5\%} & \textbf{94.1\%} \\
                    \bottomrule
                \end{tabular}
            \end{table}
            
            \vspace{0.2cm}
            \textbf{Ergebnis:} Das Wochenende ist signifikant pünktlicher (+1.6\%). Das System läuft kapazitativ am Limit.
        \end{column}
    \end{columns}
\end{frame}

% --- Folie 9: Fazit ---
\begin{frame}{Fazit \& Management Summary}
    \begin{block}{Antwort auf die Forschungsfrage}
        \textbf{Nein, grosse Bahnhöfe retten uns nicht.} Die operative Komplexität (Passagierwechsel, Kreuzungen) frisst die theoretischen Pufferzeiten auf.
    \end{block}
    
    \vspace{0.5cm}
    
    \textbf{Handlungsempfehlungen:}
    \begin{enumerate}
        \item \textbf{Fokus auf Hubs:} Massnahmen zur Pünktlichkeitssteigerung müssen in den grossen Knoten ansetzen (z.B. schnellere Abfertigung), nicht auf der Strecke.
        \item \textbf{Abendspitze entlasten:} Das System erholt sich tagsüber nicht; Taktverdichtungen am Abend sind riskant.
    \end{enumerate}
\end{frame}

% --- Folie 10: Ende ---
\begin{frame}
    \centering
    \Huge \textbf{Vielen Dank!}
    
    \vspace{1cm}
    \normalsize
    Fragen \& Diskussion
\end{frame}

\end{document}